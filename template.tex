%%%%%%%%%%%%%%%%%%%%%%%%%%%%%%%%%%%%%%%%%
% Twenty Seconds Resume/CV
% LaTeX Template
% Version 1.0 (14/7/16)
%
% Original author:
% Carmine Spagnuolo (cspagnuolo@unisa.it) with major modifications by 
% Vel (vel@LaTeXTemplates.com) and Harsh (harsh.gadgil@gmail.com)
%
% License:
% The MIT License (see included LICENSE file)
%
%%%%%%%%%%%%%%%%%%%%%%%%%%%%%%%%%%%%%%%%%

%----------------------------------------------------------------------------------------
%	PACKAGES AND OTHER DOCUMENT CONFIGURATIONS
%----------------------------------------------------------------------------------------

\documentclass[letterpaper]{twentysecondcv} % a4paper for A4

% Command for printing skill overview bubbles
\newcommand\skills{ 
~
	\smartdiagram[bubble diagram]{
        \textbf{Software}\\\textbf{Engineering},
        \textbf{Serverless}\\\textbf{Architectures},
        \textbf{Cloud}\\\textbf{Computing},
        \textbf{Big Data}\\\textbf{Platforms},
        \textbf{AWS \& Azure}\\\textbf{Platform},
        \textbf{Approximate}\\\textbf{Computing},
        \textbf{IoT}\\\textbf{Platforms}
    }
}

% Programming skill bars
\programming{{GO $\textbullet$ PHP/ 0.8},{C++ $\textbullet$ Matlab  $\textbullet$ JavaScript $\textbullet$ C\# / 2.5}, {Spark $\textbullet$ SQL $\textbullet$ \large \LaTeX $\textbullet$ Shell / 4}, {C $\textbullet$ Python $\textbullet$ Java / 5.8}}

% Projects text
% \education{
% Italian: native;\\
% English:
% }

\lang{
\textbf{Italian}: native.\\
\textbf{English}: high intermediate level (B2).
}

\aboutMe{
    Currently a student of Computer Science at Politecnico di Milano, I'm going to defend my thesis on April 29th 2020. My current research activity focuses on battery-less IoT platforms and Approximate Computing. I'm in\-ter\-est\-ed in emerging computing ar\-chi\-tec\-ture such as serverless or more in general microservices. I am getting familiar with these fields during my current contribution to Filaindiana.it, where I'm facing off the challenges of the team working, and of some business dynamics.\\

    Ex-basketball player, currently volleyball player.
}

\cvInterest{{\huge\faPlane \hspace{0.5em} \faFutbolO \hspace{0.5em} \faGamepad \hspace{0.5em} \faBeer \hspace{0.5em} \faTelevision}}
\cvExp{
    \begin{itemize}
        \item Clown Therapy \href{http://clowns.it}{@ClownOneItalia}
        \item Presidente di seggio (Returning officer)
    \end{itemize}
}

%----------------------------------------------------------------------------------------
%	 PERSONAL INFORMATION
%----------------------------------------------------------------------------------------
% If you don't need one or more of the below, just remove the content leaving the command, e.g. \cvnumberphone{}

\cvname{Francesco\\\textbf{Cerizzi}}% Your name
\cvjobtitle{ Software Engineer } % Job
% title/career
\profilepic{img/profile.png}
\cvlinkedin{francesco-cerizzi}
\cvgithub{Cerrix}
\cvnumberphone{drop me a message \faPaperPlaneO} % Phone number
\cvsite{} % Personal website
\cvmail{drop me a message \faPaperPlaneO} % Email address
\cvhome{Italy}
\cvdate{Last updated 10 April, 2020}
%----------------------------------------------------------------------------------------

\begin{document}

\makeprofile % Print the sidebar
%----------------------------------------------------------------------------------------
%    Education
%----------------------------------------------------------------------------------------

\section{Education}
\begin{twenty}
	\twentyitem
	{2017 -}
	{Present}
	{Master of Science in Computer Science and Engineer}
	{}
	{
	(Thesis dissertation on April 29th 2020)\\ \\
	{
	\textbf{Thesis}: Approximate Energy-aware Framework to Support Intermittent Computing
	}
	}
	{\footnotesize\rightline{\href{https://www.polimi.it/en/}{Politecnico di Milano}}}

	\twentyitem
	{2014 - 2016}
	{}
	{Laurea triennale in Ingegneria Informatica}
	{}
	{}
	{\footnotesize\rightline{\href{https://www.polimi.it/en/}{Politecnico di Milano}}}
\end{twenty}


%----------------------------------------------------------------------------------------
%    RESEARCH
%----------------------------------------------------------------------------------------
\section{Research}
\begin{twenty}
	\twentyitem
	{2017 - 2019}
	{}
	{Approximate Energy-aware Framework to Support Intermittent Computing.}
	{}
	{
		A new class of green and eco-friendly IoT devices, that operate without batteries or tethered power line, is gaining momentum. Indeed batteries have failed to keep pace with trends in chip and circuit and require a lot of maintenance, furthermore exhausted batteries have an impact from an environmental point of view. These devices harvest the energy needed for the computation from the environment, and as a consequence, they have strict energy constraints. My research provides a framework that on these devices achieves, by using approximate computing techniques, the execution of an application inside a user-defined energy budget.
			{
			}
	}
	{\footnotesize\rightline{\href{https://www.dropbox.com/s/wqvtatwrx4cd07j/2020_04_Cerizzi.pdf?dl=0}{\textcolor{yt}{\faFilePdfO}}\hspace{0.6em}\href{http://www.neslab.it}{NESLab}}}
	\\
	\twentyitem
	{2018 - 2019}
	{}
	{A Runtime Resource Management Policy for OpenCL Workloads on Heterogeneous Multicores.}
	{}
	{
	Nowadays, runtime workload distribution and resource tuning for heterogeneous multicores running multiple OpenCL applications is still an open quest. We propose an adaptive scheduling policy capable at identifying an optimal working point for an unknown multiprogrammed OpenCL workload without using any design-time application profiling or analysis. The approach compared against a design-time optimization strategy demonstrates to be effective in converging to a solution guaranteeing required performance while minimizing power consumption and maximum temperature. Our scheduler, thanks to a temperature predictor module, avoids frequent temperature changes that are a threat to the dependability of a computer system. We presented and published our results at \href{https://past.date-conference.com}{DATE 2019}.
	{
	}
	}
	{\footnotesize\rightline{\href{https://re.public.polimi.it/retrieve/handle/11311/1101590/361712/paper-ieee.pdf}{\textcolor{yt}{\faFilePdfO}}\hspace{0.6em}\href{https://past.date-conference.com}{DATE 2019}}}
	\\
	\twentyitem
	{2017 - 2018}
	{}
	{Ahù, a Nomadic Smart Space for Children’s Playful Learning and Inclusion.}
	{}
	{I worked in a team that produced Ahù, a technological featured `totem' to improve learning collaboration and inclusion capabilities in kindergartens and primary schools. Ahù achieves its goals by providing a combination of features that make it unique to the currently available technologies for learning: it has a totem shape, it supports multisensory stimuli by communicating through voice, lights and multimedia contents that are projected on two different fields on the floor and it allows cooperative and competitive games.
			{
			}
	}
	{\footnotesize\rightline{\href{https://www.youtube.com/watch?v=-BG5eRPgIzk&feature=youtu.be}{\textcolor{yt}{\faYoutubePlay}}\hspace{0.6em}\href{https://i3lab.polimi.it}{I3lab}}}
	\\
\end{twenty}

%----------------------------------------------------------------------------------------
%    PUBLICATIONS
%----------------------------------------------------------------------------------------

\section{Publications}
D. Angioletti, F. Bertani, C. Bolchini, \textbf{F. Cerizzi} and A. Miele, “A Runtime Resource Management Policy for OpenCL Workloads on Heterogeneous Multicores” 2019 Design, Automation \& Test in Europe Conference \& Exhibition (DATE), Florence, Italy, 2019, pp. 1385-1390.\\ {\footnotesize\rightline{\href{https://re.public.polimi.it/retrieve/handle/11311/1101590/361712/paper-ieee.pdf}{\textcolor{yt}{\faFilePdfO}}}}
\vspace{2mm}

%----------------------------------------------------------------------------------------
%    PAGE 2
%----------------------------------------------------------------------------------------

\newpage
\makenewprofile

%----------------------------------------------------------------------------------------
%	 WORK EXPERIENCE
%----------------------------------------------------------------------------------------

\section{Work Experience}

\begin{twenty} % Environment for a list with descriptions
	\twentyitem
	{2020 -}
	{Present}
	{Contributor of Filaindiana.it}
	{\href{https://www.filaindiana.it}{Filaindiana.it}}
	{}
	{Filaindiana.it is a non-profit and crowd-sourced project that aims to help people during the lockdown due to \textbf{COVID-19}. Filaindiana.it is a web application that reports the length of the supermarket queues by using user-provided data. I developed and deployed some back-end Python functions in a serverless architecture (AWS Lambda) that uses a NoSQL database (Amazon DynamoDB) hosted on Amazon Web Services.
		\begin{description}
			\item[Tools:] Amazon DynamoDB, AWS Labmda, Serverless framework (open source), Python.
		\end{description}}
	\\
	\twentyitem
	{2016 -}
	{Present}
	{Private Tuitions}
	{}
	{}
	{I give private lessons of maths, physics, and computer science from high school to university.}
	\\
	\twentyitem
	{2019}
	{}
	{Paddock Club Staff Member}
	{\href{https://tickets.formula1.com/it/h-formula1-hospitality}{Monza's Paddock club}}
	{}
	{I worked as a staff member of the Paddock Club during the Italian Grand Prix of 2019.}
	\\
	\twentyitem
	{2018 -}
	{2019}
	{Contributor of EUBra-BIGSEA}
	{\href{https://www.eubra-bigsea.eu}{EUBra-BIGSEA}}
	{}
	{EUBra-BIGSEA is a project funded in the third coordinated call Europe – Brazil focused on the development of advanced QoS services for Big Data applications. I worked in a small team that provided a Spark solution paired with a REST API to offer a \href{https://www.eubra-bigsea.eu/node/301}{data quality as-a-service solution}. Specifically I extended an \href{https://www.politesi.polimi.it/handle/10589/134468}{existing Spark solution} with data quality dimensions on a generic dataset. Furthermore I provided a REST API, powered by Flask, to submit a data quality assessment request, check status and collect results.
		\begin{description}
			\item[Tools:] Apache Spark, Flask, Python, Docker.
		\end{description}}
	%\twentyitem{<dates>}{<title>}{<location>}{<description>}
\end{twenty}
\end{document}
