%%%%%%%%%%%%%%%%%%%%%%%%%%%%%%%%%%%%%%%%%
% Twenty Seconds Resume/CV
% LaTeX Template
% Version 1.0 (14/7/16)
%
% Original author:
% Carmine Spagnuolo (cspagnuolo@unisa.it) with major modifications by 
% Vel (vel@LaTeXTemplates.com) and Harsh (harsh.gadgil@gmail.com)
%
% License:
% The MIT License (see included LICENSE file)
%
%%%%%%%%%%%%%%%%%%%%%%%%%%%%%%%%%%%%%%%%%

%----------------------------------------------------------------------------------------
%	PACKAGES AND OTHER DOCUMENT CONFIGURATIONS
%----------------------------------------------------------------------------------------

\documentclass[letterpaper]{twentysecondcvit} % a4paper for A4

% Command for printing skill overview bubbles
\newcommand\skills{ 
~
	\smartdiagram[bubble diagram]{
        \textbf{Ingegneria del}\\\textbf{software},
        \textbf{Architetture}\\\textbf{Serverless},
        \textbf{Cloud}\\\textbf{Computing},
        \textbf{Piattaforme}\\\textbf{big data},
        \textbf{Piattaforme}\\\textbf{AWS \& Azure},
        \textbf{Calcolo}\\\textbf{approssimato},
        \textbf{Piattaforme}\\\textbf{IoT}
    }
}

% Programming skill bars
\programming{{GO $\textbullet$ PHP/ 0.8},{C++ $\textbullet$ Matlab  $\textbullet$ JavaScript $\textbullet$ C\# / 2.5}, {Spark $\textbullet$ SQL $\textbullet$ \large \LaTeX $\textbullet$ Shell / 4}, {C $\textbullet$ Python $\textbullet$ Java / 5.8}}

% Projects text
% \education{
% Italian: native;\\
% English:
% }

\lang{
\textbf{Italian}: madrelingua.\\
\textbf{English}: B2.
}

\aboutMe{
    Currently a student of Computer Science at Politecnico di Milano, I'm going to defend my thesis on April 29th 2020. My current research activity focuses on battery-less IoT platforms and Approximate Computing. I'm in\-ter\-est\-ed in emerging computing ar\-chi\-tec\-ture such as serverless or more in general microservices. I am getting familiar with these fields during my current contribution to Filaindiana.it, where I'm facing off the challenges of the team working, and of some business dynamics.\\

    Ex-basketball player, currently volleyball player.
}

\cvInterest{{\huge\faPlane \hspace{0.5em} \faFutbolO \hspace{0.5em} \faGamepad \hspace{0.5em} \faBeer \hspace{0.5em} \faTelevision}}
\cvExp{
    \begin{itemize}
        \item Clown terapia \href{http://clowns.it}{@ClownOneItalia}
        \item Presidente di seggio
    \end{itemize}
}

%----------------------------------------------------------------------------------------
%	 PERSONAL INFORMATION
%----------------------------------------------------------------------------------------
% If you don't need one or more of the below, just remove the content leaving the command, e.g. \cvnumberphone{}

\cvname{Francesco\\\textbf{Cerizzi}}% Your name
\cvjobtitle{ Ingegnere del software } % Job
% title/career
\profilepic{img/profile.png}
\cvlinkedin{francesco-cerizzi}
\cvgithub{Cerrix}
\cvnumberphone{(+39) Drop me an email \faPaperPlaneO} % Phone number
\cvsite{} % Personal website
\cvmail{fracerri94@gmail.com} % Email address
\cvhome{Italia}
\cvdate{Aggiornato il \shortdate\today}
%----------------------------------------------------------------------------------------

\begin{document}

\makeprofile % Print the sidebar
%----------------------------------------------------------------------------------------
%    Education
%----------------------------------------------------------------------------------------

\section{Education}
\begin{twenty}
	\twentyitem
	{2017 -}
	{Presente}
	{Master of Science in Computer Science and Engineer}
	{}
	{
	(Discussione tesi di laurea il 29 Aprile 2020)\\ \\
	{
	\textbf{Thesis}: Approximate Energy-aware Framework to Support Intermittent Computing
	}
	}
	{\footnotesize\rightline{\href{https://www.polimi.it/}{Politecnico di Milano}}}

	\twentyitem
	{2014 - 2016}
	{}
	{Laurea triennale in Ingegneria Informatica}
	{}
	{}
	{\footnotesize\rightline{\href{https://www.polimi.it/}{Politecnico di Milano}}}
\end{twenty}


%----------------------------------------------------------------------------------------
%    RESEARCH
%----------------------------------------------------------------------------------------
\section{Research}
\begin{twenty}
	\twentyitem
	{2017 - 2019}
	{}
	{Approximate Energy-aware Framework to Support Intermittent Computing.}
	{}
	{
	Oggigiorno, una nuova tipologia di dispositivi IoT, che operano senza l'utilizzo di batterie, sta guadagnando importanza. Infatti, il progresso delle batterie non ha mantenuto il passo di quello informatico in termini di efficienza e di dimensioni; inoltre le batterie richiedono molta manutenzione e la loro dismissione ha un grande impatto ambientale. Questi dispositivi si alimentano assorbendo energia dall'ambiente che li circonda, per questo motivo hanno vincoli energetici molto stringenti. Il mio lavoro si concentra sul fornire un framework per questi dispostivi che, utilizzando tecniche di calcolo approssimato, riesce ad eseguire un'applicazione all'interno di un budget energetico richiesto dall'utente.
	{
	}
	}
	{\footnotesize\rightline{\href{https://www.dropbox.com/s/wqvtatwrx4cd07j/2020_04_Cerizzi.pdf?dl=0}{\textcolor{yt}{\faFilePdfO}}\hspace{0.6em}\href{http://www.neslab.it}{NESLab}}}
	\\
	\twentyitem
	{2018 - 2019}
	{}
	{A Runtime Resource Management Policy for OpenCL Workloads on Heterogeneous Multicores.}
	{}
	{
	La ricerca sulla distribuzione del carico di lavoro a runtime e sull’ottimizzazione delle risorse in sistemi multicore che eseguono applicazioni OpenCL eterogenee è tutt'oggi in corso. Proponiamo una politica di scheduling adattiva in grado di identificare il punto di lavoro ottimale per un carico di lavoro OpenCL non noto, senza effettuare nessuna profilazione o analisi in fase di progettazione. Il nostro approccio si dimostra, rispetto ad un approccio eseguito in fase di progettazione, efficace nel convergere in una soluzione che garantisce le prestazioni richieste, minimizzando il consumo di energia e la temperatura massima. Il nostro scheduler, grazie a un predittore di temperatura, riesce ad evitare frequenti cambi di temperatura; essi infatti sono una delle maggiori minacce per l'affidabilità di un sistema informatico. Abbiamo presentato e pubblicato i nostri risultati a \href{https://past.date-conference.com}{DATE 2019}.
	{
	}
	}
	{\footnotesize\rightline{\href{https://re.public.polimi.it/retrieve/handle/11311/1101590/361712/paper-ieee.pdf}{\textcolor{yt}{\faFilePdfO}}\hspace{0.6em}\href{https://past.date-conference.com}{DATE 2019}}}
	\\
	\twentyitem
	{2017 - 2018}
	{}
	{Ahù, a Nomadic Smart Space for Children’s Playful Learning and Inclusion.}
	{}
	{
		Ho lavorato in un team che ha prodotto Ahù, un `totem' tecnologico atto a migliorare la collaborazione e le capacità di inclusione fra bambini di età compresa tra i 3 e gli 8 anni. Ahù raggiunge i suoi obbiettivi alternando attività ludiche e di storytelling, facendo uso di una serie di apparati tecnologici tra cui video proiettori a corto raggio e sensori per il tracciamento dei movimenti.
			{
			}
	}
	{\footnotesize\rightline{\href{https://www.youtube.com/watch?v=-BG5eRPgIzk&feature=youtu.be}{\textcolor{yt}{\faYoutubePlay}}\hspace{0.6em}\href{https://i3lab.polimi.it}{I3lab}}}
	\\
\end{twenty}

%----------------------------------------------------------------------------------------
%    PUBLICATIONS
%----------------------------------------------------------------------------------------

\section{Publications}
D. Angioletti, F. Bertani, C. Bolchini, \textbf{F. Cerizzi} and A. Miele, “A Runtime Resource Management Policy for OpenCL Workloads on Heterogeneous Multicores” 2019 Design, Automation \& Test in Europe Conference \& Exhibition (DATE), Florence, Italy, 2019, pp. 1385-1390.\\ {\footnotesize\rightline{\href{https://re.public.polimi.it/retrieve/handle/11311/1101590/361712/paper-ieee.pdf}{\textcolor{yt}{\faFilePdfO}}}}
\vspace{2mm}

%----------------------------------------------------------------------------------------
%    PAGE 2
%----------------------------------------------------------------------------------------

\newpage
\makenewprofile

%----------------------------------------------------------------------------------------
%	 WORK EXPERIENCE
%----------------------------------------------------------------------------------------

\section{Work Experience}

\begin{twenty} % Environment for a list with descriptions
	\twentyitem
	{2020 -}
	{Presente}
	{Collaboratore di Filaindiana.it}
	{\href{https://www.filaindiana.it}{Filaindiana.it}}
	{}
	{
		Filaindiana.it è un progetto non-profit e crowd-sourced che vuole essere d'aiuto alla popolazione durante il periodo di lockdown dovuto al diffondersi di \textbf{COVID-19}. Filaindiana.it è un'applicazione web che, basandosi sulle segnalazione dei suoi utenti, permette di avere una stima della coda presente nei vari supermercati d'Italia. Mi sono occupato dello sviluppo e messa in opera su una architettura serverless (AWS Lambda) di alcune delle funzionalità di back-end del progetto.
		\begin{description}
			\item[Tools:] Amazon DynamoDB, AWS Labmda, Serverless framework (open source), Python.
		\end{description}}
	\\
	\twentyitem
	{2016 -}
	{Present}
	{Private Tuitions}
	{}
	{}
	{I give private lessons of maths, physics, and computer science from high school to university.}
	\\
	\twentyitem
	{2019}
	{}
	{Paddock Club Staff Member}
	{\href{https://tickets.formula1.com/it/h-formula1-hospitality}{Monza's Paddock club}}
	{}
	{I worked as a staff member of the Paddock Club during the Italian Grand Prix of 2019.}
	\\
	\twentyitem
	{2018 -}
	{2019}
	{Contributor of EUBra-BIGSEA}
	{\href{https://www.eubra-bigsea.eu}{EUBra-BIGSEA}}
	{}
	{EUBra-BIGSEA is a project funded in the third coordinated call Europe – Brazil focused on the development of advanced QoS services for Big Data applications. I worked in a small team that provided a Spark solution paired with a REST API to offer a \href{https://www.eubra-bigsea.eu/node/301}{data quality as-a-service solution}. Specifically I extended an \href{https://www.politesi.polimi.it/handle/10589/134468}{existing Spark solution} with data quality dimensions on a generic dataset. Furthermore I provided a REST API, powered by Flask, to submit a data quality assessment request, check status and collect results.
		\begin{description}
			\item[Tools:] Apache Spark, Flask, Python, Docker.
		\end{description}}
	%\twentyitem{<dates>}{<title>}{<location>}{<description>}
\end{twenty}
\end{document}
